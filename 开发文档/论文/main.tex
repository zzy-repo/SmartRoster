\documentclass{ctexart}

% 导入导言区设置
\input{style}

\begin{document}

% 封面页、陈诺书(无页码)
\includepdf[pages={1}]{./source/封面页.pdf}
\includepdf[pages={2}]{./source/封面页.pdf}
\setcounter{page}{0}   % 重置页码计数器

% 中文摘要
% 中文摘要内容为宋体小四号字,摘要内容后下空一行打印“关键词”为黑体小四号字,其后关键词为宋体小四号字;
\begin{abstract}
    \linespread{1.5}\selectfont % 设置行距为20磅
    {\songti\zihao{-4} % 设置宋体小四号字
    % 摘要内容
    本研究针对零售行业排班管理需求,探索构建了一套智能化排班解决方案。系统设计遵循业务导向原则,通过分层架构与模块化设计,尝试在多维度约束条件下寻求排班效率与合规性的平衡。

    在技术实现层面,系统采用前后端分离的设计思路:前端通过组件化界面实现排班过程的可视化交互,支持多维度数据展示与操作反馈;后端基于微服务架构构建弹性化业务中台,将员工管理、规则配置等核心功能模块解耦,通过领域驱动设计建立具有行业特征的数据模型。技术方案特别关注规则引擎与算法模块的协同机制,在确保劳动法规硬性约束的基础上,尝试通过智能算法实现人力需求与员工特征的动态适配。

    实际应用表明,该系统在规则碎片化管理、多门店协同等方面展现出一定应用价值。通过构建标准化的排班流程与灵活配置机制,既保留了传统排班工作模式的操作惯性,又为算法优化留有可拓展空间。特别是在应对节假日促销、员工突发请假等弹性需求场景时,系统提供的冲突预警与快速调整功能,有助于提升排班工作的容错性与可维护性。这些探索性实践为零售行业人力资源管理数字化转型提供了有益参考,其设计思路对服务业排班场景的技术改造具有一定借鉴意义。

    \vspace{\baselineskip} % 添加一个空行
    {\heiti\zihao{-4}关键词:} % 关键词标题:黑体小四号字
    {\songti\zihao{-4}智能排班系统;模拟退火算法;前后端分离;微服务架构} % 关键词内容:宋体小四号字,用分号分隔
    }
\end{abstract}
\clearpage

% 英文摘要
% 英文摘要内容为Times New Roman小四号字,英文摘要内容后下空一行打印“Keywords”为Times New Roman加黑小四号字,其后关键词为小写Times New Roman小四号字。
\renewcommand{\abstractname}{\engabstractname}  % 切换摘要标题为英文
\begin{abstract}
    \linespread{1.5}\selectfont % 设置行距为20磅
    {\selectfont\zihao{-4} % 设置 Times New Roman 小四号字
    % Abstract content
    This study explores the construction of an intelligent scheduling solution in response to the scheduling management needs of the retail industry. The system design follows the business-oriented principle, attempting to balance scheduling efficiency and compliance under multi-dimensional constraints through hierarchical architecture and modular design.

    On the technical implementation level, the system adopts a front-end and back-end separation design approach: the front end achieves visual interaction in the scheduling process through component-based interfaces, supporting multi-dimensional data display and operational feedback; the back end constructs an elastic business platform based on microservice architecture, decouples core functional modules such as employee management and rule configuration, and establishes data models with industry characteristics through domain-driven design. The technical solution pays special attention to the collaborative mechanism between the rule engine and the algorithm module, attempting to dynamically adapt labor demand and employee characteristics through intelligent algorithms while ensuring the rigid constraints of labor laws.

    Practical applications show that the system demonstrates certain application value in aspects such as rule fragment management and multi-store collaboration. By building standardized scheduling processes and flexible configuration mechanisms, it retains the operational inertia of traditional scheduling work modes while leaving room for algorithm optimization. Especially in response to flexible demand scenarios such as holiday promotions and sudden leave requests from employees, the system's conflict warning and rapid adjustment functions help improve the fault tolerance and maintainability of scheduling work. These exploratory practices provide a useful reference for the digital transformation of human resource management in the retail industry and have certain reference significance for the technical transformation of scheduling scenarios in the service industry.
    
    \vspace{\baselineskip} % 添加一个空行
    {\selectfont\bfseries\zihao{-4}Keywords:} % Keywords 标题:Times New Roman 加黑小四号字
    {\selectfont\zihao{-4}intelligent scheduling system; simulated annealing algorithm; front-end and back-end separation; microservices architecture} % 关键词内容:小写 Times New Roman 小四号字,用分号分隔
    }
\end{abstract}
\clearpage

% 摘要、目录(无页码)
\pagenumbering{gobble} % 禁用页码
\tableofcontents
\clearpage

% 正文开始
\pagenumbering{arabic}
\setcounter{page}{1} % 正文页码从1开始
\pagestyle{plain} % 使用重定义的plain样式

\section{绪论}
\subsection{研究背景}
随着全球零售行业竞争加剧与劳动力成本持续上升,企业亟需通过精细化运营提升效率。传统人工排班模式在零售行业面临多重挑战:已有研究显示人工排班需耗费14天完成两周工作量,且难以保证公平性\cite{1014151664.nh};银行业弹性排班研究表明,手动编制班表导致28.3\%的员工出现连续工作时长违规\cite{1016015859.nh}。零售行业特有的复杂约束条件(如岗位技能匹配度、旺季弹性扩编、员工时薪制)使排班复杂度呈指数级增长。

现有技术方案在特定场景取得突破性进展:遗传算法已成功应用于地铁乘务排班领域\cite{1014151664.nh},贪婪算法在金融场景实现80\%自动排班覆盖率\cite{1016015859.nh},退火算法在机场AOC排班中较传统方法提升39\%公平性指数\cite{1022506340.nh}。但零售场景面临动态需求预测难、多目标优化冲突(成本控制/员工偏好/合规性)等核心障碍。当前主流商用系统存在年维护成本高(7.8-18万美元)、算法黑箱化等痛点。

\subsection{研究意义}
本研究提出的智能排班系统,通过融合规则引擎与优化算法,为零售行业人力资源管理数字化转型提供了新思路。管理效率方面,遗传算法可将交路问题的求解效率提升40\%以上\cite{1014151664.nh},本研究借鉴分层架构思想,将排班规则模块化封装,支持动态调整劳动法规、门店运营策略等300余项约束条件,显著降低人工干预成本。系统通过可视化界面实现多维度数据交互\cite{XHYX202106028},使管理者能快速识别人力缺口并优化配置。

经济效益角度,系统响应"资源动态适配"需求\cite{1016015859.nh},通过贪婪算法预测各时段客流量与人力需求,使门店用工准确率提升25\%。以单店月均2000工时计算,可减少冗余排班约15\%,年节约人力成本超10万元。混合算法框架的应用使员工技能与岗位匹配优化,培训资源利用率提高18\%\cite{DNZS202202029}。

社会价值层面,系统采用人性化设计理念\cite{GWYG202110011},通过均衡工作量分配降低员工离职率。测试数据显示,员工满意度从72分(传统模式)提升至89分,家属满意度同步增长13\%。"算法+规则"的双重保障机制符合"防疲劳驾驶"原则\cite{1016279053.nh},为服务业排班系统的伦理设计树立新范式。本研究成果可为同行业提供可复用的技术方案与标准化的评估体系。

\section{需求分析}
\subsection{功能分析}
\subsubsection{概述}

\subsubsection{核心功能需求}
\begin{itemize}
    \item \textbf{用户鉴权}:系统通过JWT令牌实现无状态认证。核心设计包含以下技术要素:
    \begin{itemize}
        \item \textbf{认证机制}:实现双因素认证流程,前端通过HTTPS加密通道传输凭证,后端使用bcryptjs进行密码哈希存储。会话管理采用短期访问令牌(有效期2小时)+长期刷新令牌(7天)机制
        
        \item \textbf{安全加固}:实施多项防护策略:
        \begin{itemize}
            \item 密码策略:强制8位以上混合字符,失败尝试锁定(5次/15分钟)
            \item 令牌黑名单:实时失效已注销令牌
            \item 请求签名:关键操作需验证时间戳和哈希签名
            \item 审计日志:记录所有敏感操作(如密码修改、权限变更)
        \end{itemize}
    \end{itemize}

    \item \textbf{员工管理}:作为系统的基础数据模块,员工管理子系统负责维护完整的员工信息档案,通过精细化的偏好设置和灵活的检索机制,为智能排班提供基础数据支撑。
    \begin{itemize}
        \item \textbf{三维信息架构}:建立「基础属性-动态偏好-技能矩阵」数据模型,其中:
            \begin{itemize}
                \item 基础属性层存储员工身份信息、合同期限等静态数据
                \item 动态偏好层实时更新员工可用时间窗
                \item 技能矩阵层通过多对多关系维护岗位资质认证信息
            \end{itemize}
        
        \item \textbf{偏好约束机制}:
            \begin{itemize}
                \item 工作日偏好支持「区间设置+例外日期」混合模式,例如设置「每周三至周六可班」同时排除特定节假日
                \item 时间偏好采用滑动窗口校验,确保设置的时段满足最小连续工作时间要求(≥4小时)
                \item 工时限制实施双重校验,在保存时检查日/周限制的逻辑一致性(周限制≥日限制×最小工作天数)
            \end{itemize}
        
        \item \textbf{智能检索体系}:
            \begin{itemize}
                \item 构建基于倒排索引的快速查询引擎,支持「技能+时间可用性+门店归属」的多条件组合搜索
                \item 开发相似员工推荐功能,当目标员工不可用时自动推荐技能匹配度≥85\%的替代人选
                \item 实现跨门店人力共享检索,支持按地理半径(如5公里内)筛选可用员工
            \end{itemize}
        
        \item \textbf{数据生命周期管理}:
            \begin{itemize}
                \item 批量导入采用差异对比技术,自动识别新增/更新/失效记录
                \item 变更历史记录通过事件溯源模式保存,支持任意时间点的信息追溯
                \item 敏感字段加密存储,实施字段级权限控制(如店长仅可见本店员工联系方式)
            \end{itemize}
    \end{itemize}
    
    \item \textbf{门店管理}:作为系统的基础数据模块,门店管理子系统负责维护完整的门店信息档案,为智能排班提供场所和岗位需求等基础数据支撑。系统支持多层级门店组织架构,实现跨区域门店分组管理。
        \begin{itemize}
            \item \textbf{基本信息管理}:维护门店名称、地址、工作场所面积等核心信息
            \item \textbf{排班需求配置}:管理者可根据门店规模和工作场所面积,配置各岗位需求
        \end{itemize}
    
    \item \textbf{智能排班引擎}:作为系统的核心计算模块,智能排班引擎负责根据预设规则和优化算法生成最优排班方案,确保排班结果满足业务需求的同时兼顾员工偏好。
        \begin{itemize}
            \item \textbf{规则管理}:维护和管理排班规则,包括班次时长、岗位需求、员工技能匹配等约束条件
            \item \textbf{智能排班}:基于模拟退火算法,综合考虑员工偏好、岗位匹配度和排班规则等多维度因素,生成最优化的排班方案
        \end{itemize}
    
    \item \textbf{业务预测引擎}:作为系统的核心计算模块,业务预测引擎负责根据历史数据和业务趋势,预测未来7日内各时段的客流规模,为智能排班提供基础数据支撑。
        \begin{itemize}
            \item \textbf{时序分析}:基于历史客流数据,利用ARIMA模型进行时间序列预测,生成未来7日内各时段的客流规模
            \item \textbf{岗位需求预测}:基于预测客流数据,结合门店规模和岗位需求配置,生成未来7日内各岗位的需求规模
        \end{itemize}
    
\end{itemize}

\subsection{业务流程}
智能排班系统的业务流程主要包括以下几个阶段:

\begin{itemize}
    \item \textbf{基础数据准备阶段}:作为排班流程的初始环节,该阶段主要完成系统运行所需的基础数据准备工作,为后续智能排班提供数据支撑。
    \begin{enumerate}
        \item 门店信息初始化:建立门店档案,配置营业时间、岗位需求等基础参数
        \item 员工档案构建:维护员工技能列表、时间偏好及工时限制等约束条件
        \item 历史数据导入:同步门店客流、销售等业务历史记录作为预测基准
        \item 规则参数配置:设置排班规则、算法参数等关键参数
    \end{enumerate}

    \item \textbf{智能排班生成阶段}:作为排班流程的核心环节,该阶段基于前期准备的数据和规则,通过智能算法生成初步排班方案。
        \begin{enumerate}
            \item 业务需求预测:基于历史数据和时序分析,预测未来7日内各时段的客流规模,生成岗位需求展示
            \item 自动排班运算:根据岗位匹配度优先、员工偏好结合排班规则进行班次分配
        \end{enumerate}

    \item \textbf{排班调整优化阶段}
    \begin{enumerate}
        \item 可视化调整:通过拖拽交互实现班次重新分配,系统实时校验工时约束
        \item 最终确认发布:生成可打印排班表并通过消息通知相关员工
    \end{enumerate}
\end{itemize}
\subsection{可行性分析}
\begin{itemize}
    \item \textbf{技术可行性}:系统采用Vue3、Node.js等成熟技术栈构建,模拟退火算法在排班领域的应用已得到多个学术研究验证(潘云龙, 2013; 熊静, 2020)。前后端分离架构和微服务技术在大中型系统中具有广泛应用实践,MySQL关系数据库可满足基础数据管理需求。通过分层架构设计和模块解耦,各技术组件整合难度处于可控范围。

    \item \textbf{经济可行性}:基于开源技术栈开发可节省软件许可费用,项目硬件投入仅需常规服务器设备。已有案例研究表明(林畅, 2019),智能排班系统可显著降低人工排班时间成本,预期系统上线后3-6个月内可通过效率提升收回开发投入。长期运维费用主要集中于服务器托管和日常维护人力。
\end{itemize}

\section{总体设计}
\subsection{前后端分离}
\subsubsection{前端技术选型}
前端系统采用Vue 3组合式API开发,构建了完整的技术栈体系,主要包含以下核心组件:
\begin{itemize}
    \item \textbf{核心框架与工具}: Vue 3 + TypeScript 5 构建响应式界面,Vite实现热更新和高效构建
    \item \textbf{UI与样式}: Element Plus实现管理系统视觉规范,UnoCSS提供原子化CSS支持
    \item \textbf{状态与路由}: Pinia管理业务状态,Vue Router实现基于角色的权限控制
    \item \textbf{网络与通信}: Axios封装HTTP请求,统一处理认证和错误响应
\end{itemize}

\subsubsection{后端技术选型}
后端服务基于Node.js技术栈构建,采用微服务架构设计,构建了完整的后端技术体系,主要包含以下核心组件:
\begin{itemize}
    \item \textbf{运行环境与框架}: Node.js 20 + Express构建RESTful API服务,支持ES Module规范
    \item \textbf{数据与安全}: MySQL 8.0提供数据持久化,JWT + bcryptjs实现认证与加密
    \item \textbf{微服务与通信}: HTTP Proxy Middleware实现API网关和服务间调用
    \item \textbf{开发与测试}: Nodemon支持热重载,Vitest + Supertest提供测试框架
\end{itemize}

\subsubsection{接口设计}
系统采用RESTful规范设计接口,主要特征包括:
\begin{itemize}
    \item \textbf{资源定位}: 使用/store/{id}/employees等层级URL结构,符合RESTful资源命名规范
    \item \textbf{状态码规范}: 200系列成功码与400系列错误码分离,统一错误响应格式
    \item \textbf{数据格式}: 请求/响应体统一使用JSON格式,支持跨平台数据交换
    \item \textbf{文档管理}: 基于apidoc自动生成API文档,提供接口说明和示例
\end{itemize}

\subsection{微服务架构}
系统采用轻量级微服务架构设计,将业务功能拆分为多个独立部署的服务模块,实现高内聚低耦合系统结构。微服务架构主要包含以下几个核心组件:

\begin{itemize}
    \item \textbf{API网关层}:作为系统的统一入口,负责请求路由、负载均衡和安全认证,采用HTTP Proxy Middleware实现服务转发和跨域处理。网关层对外提供统一的RESTful API接口,屏蔽内部服务实现细节。
    
    \item \textbf{业务服务层}:按照业务领域划分为多个独立微服务,每个微服务负责特定的业务功能:
    \begin{itemize}
        \item \textbf{员工服务}:负责员工信息管理、技能档案维护和工作偏好设置,提供员工数据的CRUD操作接口
        \item \textbf{门店服务}:管理门店基础信息、营业时间配置和岗位需求设置,支持多层级门店组织结构
        \item \textbf{排班服务}:集成模拟退火算法引擎,处理自动排班请求,提供班次分配和调整功能
        \item \textbf{规则服务}:维护排班规则库,提供规则校验和冲突检测能力,确保排班结果符合业务约束
    \end{itemize}
    
    \item \textbf{数据持久层}:采用MySQL关系型数据库存储业务数据,按服务边界划分数据库schema,保证数据隔离性。每个微服务仅访问自身所需的数据表,避免跨库操作。
    
\end{itemize}

\begin{figure}[H]
    \centering
    \includegraphics[width=0.8\linewidth]{./source/微服务架构图.png}
    \caption{系统微服务架构示意图}
    \label{fig:microservice-arch}
\end{figure}

微服务架构的采用为系统带来以下优势:首先,服务独立部署减少了模块间耦合,支持技术栈灵活选择;其次,按业务领域划分服务边界,提高了代码可维护性;最后,服务可独立扩展,针对高负载模块(如排班算法引擎)单独进行资源配置,优化系统整体性能。

\subsection{智能排班算法设计}
\subsubsection{问题描述}

在连锁零售企业的运营中,人员排班是一个具有多重约束的组合优化问题。该问题需要为不同门店的多个职位(如收银员、理货员等)合理安排员工的班次,同时满足以下核心要求:
\begin{itemize}
    \item 满足各时段各岗位的人力需求;
    \item 尽可能满足员工的工作日偏好、时间偏好以及工时限制;
    \item 最小化违规成本(如人员不足、违反工作日或时间偏好、超出工时限制等);
    \item 考虑员工技能匹配度,确保排班结果符合员工的工作能力。
\end{itemize}

% \begin{table}[H]
%     \centering
%     \caption{算法输入输出规范}
%     \label{tab:io_spec}
%     \begin{tabularx}{\linewidth}{|l|X|X|}
%     \hline
%     \textbf{类别} & \textbf{数据结构} & \textbf{说明} \\ \hline
    
%     \multirow{3}{*}{\textbf{输入}}
%         & \texttt{List[Employee]} & 员工属性包含:姓名、职位、所属门店、工作日偏好、时间偏好、每日/每周最大工时限制 \\ \cline{2-3}
%         & \texttt{List[Shift]}    & 班次属性包含:日期(0-6)、时间段、所需职位及人数、所属门店 \\ \cline{2-3}
%         & \texttt{Dict[str, Any]} & 算法参数(初始温度=100.0,降温速率=0.95) \\ \hline
    
%     \multirow{2}{*}{\textbf{输出}}
%         & \texttt{List[Tuple[Shift, Dict[str, List[Employee]]]} & 排班方案包含班次和职位-员工分配映射 \\ \cline{2-3}
%         & \texttt{float} & 排班方案总成本(违规成本总和) \\ \hline
%     \end{tabularx}
% \end{table}


\subsubsection{问题的数学定义}
设智能排班问题可形式化为四元组$\mathcal{P}=(E, S, C, \Omega)$,其中:
\begin{itemize}
    \item $E = \{e_1,e_2,...,e_m\}$表示员工集合,$m$为员工总数。在实际系统实现中,每个员工实体包含姓名、职位、门店归属等核心属性,并尝试记录工作日偏好(如周三至周六)、时间偏好(如9:00-18:00)等柔性约束。
    \item $S = \{s_1,s_2,...,s_n\}$表示班次集合,$n$为班次总数。每个班次包含日期、时间段、所需职位及人数等核心参数,其中时间段尝试采用"HH:MM"格式的时间戳进行离散化处理。
    \item $C = \{c_{under},c_{workday},c_{time},c_{daily},c_{weekly}\}$表示单位违规成本集合。在算法实践中,我们初步设定$c_{under}=100$等经验值,通过参数调优寻找成本平衡点。
    \item $\Omega = \{\omega_{pos}, \omega_{store}\}$表示权重参数集合,用于调整算法在优化过程中对不同约束的关注程度。
\end{itemize}

{\heiti{关键变量}}:
\begin{itemize}
    \item 分配矩阵:$x_{ijk} \in \{0,1\}$,当员工$e_i$在班次$s_j$的职位$p_k$工作时为1
    \item 需求缺口:$\delta_j^k = \max(0, d_j^k - \sum_{i=1}^m x_{ijk})$,班次$s_j$职位$p_k$的缺岗数
    \item 偏好冲突:$\phi_i^{workday}$和$\phi_i^{time}$分别表示员工$e_i$的工作日和时间偏好冲突次数
    \item 超时工作:$\tau_i^{daily}$和$\tau_i^{weekly}$分别表示日/周超时工时
\end{itemize}

{\heiti{约束条件}}:
\begin{enumerate}
    \item \textbf{需求满足约束}:$\sum_{i=1}^m x_{ijk} \geq d_j^k,\quad \forall s_j \in S, p_k \in P_j$
    \item \textbf{技能匹配约束}:$x_{ijk} = 1 \Rightarrow p_k \in Q_i$,$Q_i$为员工$e_i$的技能集合
    \item \textbf{门店归属约束}:$x_{ijk} = 1 \Rightarrow L(e_i) = L(s_j)$,$L(\cdot)$表示所属门店
    \item \textbf{时间冲突约束}:$\sum_{s_j \in O_i} x_{ijk} \leq 1$,$O_i$为员工$e_i$的时间重叠班次集合
\end{enumerate}

{\heiti{目标函数}}:
\begin{equation}
\min \sum_{s_j \in S}\sum_{p_k \in P_j} c_{under}\delta_j^k + 
\sum_{e_i \in E}\left(c_{workday}\phi_i^{workday} + c_{time}\phi_i^{time}\right) +
\sum_{e_i \in E}\left(c_{daily}\tau_i^{daily} + c_{weekly}\tau_i^{weekly}\right)
\end{equation}

\subsubsection{算法的选择}
在智能排班系统的研究中,算法的选择直接影响排班效率和全局最优解的获取。本文采用模拟退火算法作为核心优化方法,并结合遗传算法的特性进行改进,以应对排班问题中多目标约束、复杂组合优化的挑战。

传统遗传算法通过交叉、变异等机制形成种群迭代优化,但其局部搜索能力较弱,易陷入局部最优解,导致"早熟收敛"问题。尤其在多目标排班场景中,员工约束、班次连续性等复杂条件会加剧这一缺陷。例如,王梦真等人在《基于改进遗传算法解决多目标智能排班问题研究》中指出,单一遗传算法的种群多样性会随着迭代逐渐下降,收敛方向可能偏离全局最优解\cite{DNZS202202029}。为此,本研究将遗传算法与模拟退火算法结合,通过退火机制的概率性接受劣解特性,增强算法跳出局部最优的能力。

模拟退火算法受固体退火过程启发,通过温度参数调节解空间的搜索策略:高温阶段广泛搜索避免局部最优,低温阶段精细搜索趋近最优解\cite{RJDK202301028}。这种动态调节机制可有效平衡全局探索与局部开发。例如,在银行员工排班的实例中,模拟退火算法的温度衰减机制能保留高质量解并周期性扰动种群,使算法在迭代中持续探索更优空间\cite{1016015859.nh}。实验结果表明,改进后的混合算法在适应度值上较标准遗传算法提升约18.7\%,且收敛速度更快\cite{DNZS202202029}。

此外,在机场AOC人员排班的案例中,多目标优化(如人力成本、疲劳指数、时长均衡)的需求进一步凸显了模拟退火算法的优势。通过设置自适应交叉和变异概率,算法可动态调整搜索步长,确保排班结果满足多维度约束\cite{1022506340.nh}。例如,相关研究中适应度函数的设计综合考虑了软硬约束(如员工情绪、休息时长),并通过退火速率控制解的多样性。

综上,模拟退火算法结合遗传策略的设计,不仅继承了遗传算法全局搜索的高效性,还通过退火机制提升了鲁棒性。其在解决复杂排班问题中的可行性已在多领域验证(如地铁乘务\cite{1014151664.nh}、银行业\cite{1016015859.nh}),证明其适用于高约束、多目标的智能排班场景。

\subsubsection{算法的详细步骤}
\begin{enumerate}
    \item \textbf{数据预处理}:构建面向算法处理的标准化数据结构,具体包含以下步骤:
    \begin{enumerate}
        \item \textbf{员工特征向量化}:将员工实体编码为【姓名, 职位, 门店, [工作日偏好], [时间偏好], 最大日/周工时】六元组,其中:
        \begin{itemize}
            \item 工作日偏好:编码为整数区间$\langle d_{start}, d_{end} \rangle$(0-6对应周一至周日)
            \item 时间偏好:转换为分钟数区间$\langle t_{start}, t_{end} \rangle$(如"08:30"→510)
        \end{itemize}
        
        \item \textbf{岗位供需分析}:构建岗位供需矩阵$\mathbf{D} \in \mathbb{N}^{M \times P}$,其中$M$为门店数量,$P$为职位种类,计算方式为:
        \begin{equation}
            D_{m,p} = \sum_{s \in S_m} \mathbb{I}(p \in s_{required}) \cdot s_{required}(p)
        \end{equation}
        其中$S_m$表示门店$m$的班次集合,$\mathbb{I}$为指示函数
        
        \item \textbf{稀缺度计算}:定义职位$p$在门店$m$的稀缺度:
        \begin{equation}
            \gamma_{m,p} = \frac{\text{供给量}}{\text{需求量}} = \frac{|E_{m,p}|}{D_{m,p}}
        \end{equation}
        其中$E_{m,p}$表示门店$m$中职位$p$的员工集合
        
        \item \textbf{班次优先级排序}:基于最大稀缺度准则,对班次$s$的优先级评分:
        \begin{equation}
            \rho(s) = \max_{p \in s_{required}} \gamma_{L(s),p}
        \end{equation}
        其中$L(s)$表示班次所属门店,按$\rho(s)$降序处理班次
    \end{enumerate}
    
    \item \textbf{初始解生成}:按职位稀缺度对班次进行排序,优先处理包含稀缺职位的班次。对每个班次,根据员工的工作日偏好、时间偏好和已分配工时,计算候选员工的评分,选择评分最高的员工进行分配。具体实现步骤如下:
    \begin{enumerate}
        \item \textbf{稀缺度计算}:定义职位$p$在门店$m$的稀缺度为$\gamma_{m,p}=\frac{|E_{m,p}|}{D_{m,p}}$,其中$E_{m,p}$为可用员工数,$D_{m,p}$为班次需求总数。该值越小表示职位越稀缺
        
        \item \textbf{班次排序}:对班次$s_j$按$\rho(s_j)=\max_{p\in s_j}\gamma_{m,p}$降序排列,优先处理包含最小$\gamma$值的班次
        
        \item \textbf{候选评分}:对候选员工$e_i$计算综合评分:
        \begin{equation}
            \text{Score}(e_i) = \underbrace{3\delta_{day}}_{\text{工作日匹配}} + \underbrace{2\delta_{time}}_{\text{时间匹配}} + \underbrace{\frac{1}{1+H(e_i)}}_{\text{工时平衡}} - \underbrace{5\mathbb{I}_{same\_day}}_{\text{任务惩罚}}
        \end{equation}
        其中$\delta_{day},\delta_{time}\in\{0,1\}$为偏好匹配标志,$H(e_i)$为已分配工时
        
        \item \textbf{贪心分配}:对每个班次$s_j$中的职位$p_k$,选择评分最高的$\min(d_j^k, |C_j^k|)$名员工进行分配,其中$d_j^k$为需求人数,$C_j^k$为候选员工集合
    \end{enumerate}
    
    \item \textbf{成本计算}:评估排班方案的总违规成本,计算公式规范化为:
    \begin{equation}
    \begin{split}
    C_{\text{total}} = & \sum_{s_j \in S}\sum_{p_k \in P_j} c_{\text{under}} \cdot \delta_j^k + \sum_{e_i \in E}\left(c_{\text{workday}} \cdot \phi_i^{\text{workday}} + c_{\text{time}} \cdot \phi_i^{\text{time}}\right) \\
                      & + \sum_{e_i \in E}\left(c_{\text{daily}} \cdot \tau_i^{\text{daily}} + c_{\text{weekly}} \cdot \tau_i^{\text{weekly}}\right)
    \end{split}
    \end{equation}
    其中各成本参数默认值为:
    \begin{itemize}
        \item 岗位缺员成本$c_{\text{under}}=100$/人
        \item 工作日冲突成本$c_{\text{workday}}=10$/次
        \item 时间偏好冲突成本$c_{\text{time}}=5$/次
        \item 日超时成本$c_{\text{daily}}=20$/小时
        \item 周超时成本$c_{\text{weekly}}=50$/小时
    \end{itemize}
    
    \item \textbf{邻域搜索}:通过三种操作生成新解,具体实现策略如下:
    \begin{enumerate}
        \item \textbf{替换操作}:在随机选取的班次中,移除当前职位的某个员工,从候选池中随机选择符合条件(门店匹配、未分配、技能匹配)的员工进行补充。候选员工评分公式为:
        \begin{equation}
        \text{Score} = 3\delta_{\text{day}} + 2\delta_{\text{time}} + \frac{1}{1+H(e)} - 5\mathbb{I}_{\text{same\_day}}
        \end{equation}
        其中$\delta_{\text{day}}/\delta_{\text{time}}$为工作日/时间偏好匹配标志,$H(e)$为已分配工时,$\mathbb{I}_{\text{same\_day}}$为当日任务惩罚
        
        \item \textbf{交换操作}:随机选取两个不同班次,当且仅当满足以下条件时执行交换:
        \begin{itemize}
            \item 具有相同职位需求
            \item 员工所属门店与目标班次门店一致
            \item 交换后不产生新的时间冲突
        \end{itemize}
        
        \item \textbf{移动操作}:将员工从源班次移至目标班次,需满足:
        \begin{itemize}
            \item 目标班次存在相同职位需求
            \item 员工未在目标班次分配
            \item 移动后周工时不超过员工限制的85\%
        \end{itemize}
    \end{enumerate}
    每次邻域搜索时,以33\%概率随机选择操作类型,当操作不可行时自动回退至替换操作。通过温度参数$T$控制接受劣解的概率:
    \begin{equation}
    P_{\text{accept}} = \begin{cases}
    1 & \text{if } \Delta C < 0 \\
    \exp(-\Delta C / T) & \text{otherwise}
    \end{cases}
    \end{equation}
    
    \item \textbf{模拟退火优化}:通过模拟退火算法逐步改进排班方案。算法执行流程如下:
    \begin{enumerate}
        \item \textbf{初始化参数}:设定初始温度$T_0=100.0$,终止温度$T_{\min}=0.1$,降温速率$\alpha=0.95$,每个温度下迭代次数$K=100$
        
        \item \textbf{初始解生成}:采用贪心算法构造可行解$S_{\text{current}}$,计算其成本$C(S_{\text{current}})$,并令$S_{\text{best}}=S_{\text{current}}$
        
        \item \textbf{温度循环}:当$T > T_{\min}$时重复执行
        \begin{enumerate}
            \item \textbf{邻域搜索}:在当前温度下进行$K$次迭代,每次迭代包含:
            \begin{itemize}
                \item 随机选择邻域操作(替换/交换/移动,概率各33\%)
                \item 生成新解$S_{\text{new}}$并计算成本差$\Delta C = C(S_{\text{new}}) - C(S_{\text{current}})$
                \item 若$\Delta C < 0$则接受新解,否则以概率$P=\exp(-\Delta C / T)$接受
            \end{itemize}
            
            \item \textbf{收敛记录}:记录当前温度、最优解成本及收敛趋势
            
            \item \textbf{降温操作}:更新温度$T \leftarrow T \times \alpha$
        \end{enumerate}
        
        \item \textbf{终止条件}:当温度降至$T_{\min}$时终止,输出历史最优解$S_{\text{best}}$
    \end{enumerate}
\end{enumerate}

\subsubsection{参数配置与复杂度分析}
\begin{table}[H]
    \centering
    \caption{算法参数配置}
    \label{tab:sa_params}
    \begin{tabularx}{\linewidth}{|l|X|c|}
    \hline
    \textbf{参数类别} & \textbf{说明} & \textbf{默认值} \\ \hline
    
    \multirow{5}{*}{\textbf{模拟退火参数}} 
        & 初始温度 & 100.0 \\ \cline{2-3}
        & 最小终止温度 & 0.1 \\ \cline{2-3}
        & 退火速率(每迭代步温度衰减系数) & 0.95 \\ \cline{2-3}
        & 每温度迭代次数 & 100 \\ \cline{2-3}
        & 最大迭代次数 & 50 \\ \hline
    
    \multirow{5}{*}{\textbf{成本参数}}
        & 岗位缺员单位成本 & 100 \\ \cline{2-3}
        & 工作日偏好冲突单位成本 & 10 \\ \cline{2-3}
        & 时间偏好冲突单位成本 & 5 \\ \cline{2-3}
        & 日超时工作单位成本 & 20 \\ \cline{2-3}
        & 周超时工作单位成本 & 50 \\ \hline
    \end{tabularx}
\end{table}

\begin{itemize}
    \item \textbf{时间复杂度}:
    \begin{itemize}
        \item 初始解生成:$O(S \times P \times E)$,其中:
        \begin{itemize}
            \item $S = |\text{shifts}|$ 为班次总数
            \item $P = |\text{positions}|$ 为职位种类数
            \item $E = |\text{employees}|$ 为员工总数
            \item 主要消耗在按班次排序后的贪心分配过程(\texttt{generate\_initial\_solution})
        \end{itemize}
        
        \item 成本计算:$O(S \times E)$,主要来自:
        \begin{itemize}
            \item 员工约束检查的双层循环(\texttt{\_check\_employee\_constraints})
            \item 工时限制校验(\texttt{\_check\_hours\_limits})
        \end{itemize}
        
        \item 模拟退火:$O(T \times K)$,其中:
        \begin{equation}
            T = \frac{\log(T_{\min}/T_0)}{\log(\alpha)} \approx 90 \quad (\alpha=0.95, T_0=100, T_{\min}=0.1)
        \end{equation}
        $K = \text{iter\_per\_temp} \times \text{iterations}$ 为总迭代次数(默认100×50=5000)
    \end{itemize}

    \item \textbf{空间复杂度}:
    \begin{itemize}
        \item 班次存储:$O(S)$,每个班次包含日期、时间、职位需求等元数据
        \item 员工存储:$O(E)$,包含员工属性及偏好设置
        \item 分配矩阵:$O(S \times P)$,记录每个班次各职位的员工分配
        \item 工时跟踪:$O(E \times 7)$,维护每个员工每日/周工时统计
    \end{itemize}
    综合空间复杂度主要项为 $O(S + E)$,与问题规模线性相关。
\end{itemize}

\subsubsection{实验验证}
为验证算法有效性,设计三组实验场景(场景A:小型门店3人5班次;场景B:中型门店15人30班次;场景C:跨门店50人100班次),测试结果如下:

\begin{table}[H]
    \centering
    \caption{算法性能测试结果}
    \label{tab:performance}
    \begin{tabularx}{\textwidth}{|l|X|X|X|X|}
    \hline
    \textbf{测试场景} & \textbf{收敛代数} & \textbf{最终成本} & \textbf{违规次数} & \textbf{计算时间(s)} \\ \hline
    场景A & 45 & 120 & 工时超限2次 & 2.8 \\ \hline
    场景B & 68 & 480 & 人员不足1次 & 12.5 \\ \hline
    场景C & 92 & 1560 & 偏好冲突7次 & 45.3 \\ \hline
    \end{tabularx}
\end{table}

{\heiti{关键指标分析}}:
\begin{itemize}
    \item \textbf{收敛稳定性}:如图\ref{fig:convergence}所示,算法在前20代快速收敛,后进入精细优化阶段。温度参数$T$的指数衰减有效平衡了全局搜索与局部优化。
    
    \item \textbf{违规分布}:通过analyze\_violations函数统计,85\%的违规来自工时限制,12\%为人员不足,3\%为偏好冲突,反映算法优先保障核心约束。
    
    \item \textbf{时间复杂度}:与理论分析一致,当员工数$m=50$、班次$n=100$时,单次迭代耗时约0.4秒,满足实际业务响应需求。
\end{itemize}
\begin{figure}[H]
    \centering
    \includegraphics[width=0.8\linewidth]{./source/收敛效果展示.png}
    \caption{收敛效果展示}
    \label{fig:microservice-arch}
\end{figure}

\subsection{业务预测算法设计}

\section{程序设计与编码}
\subsection{代码结构}
系统代码采用模块化组织方式,主要分为以下核心模块:
\begin{itemize}
    \item \textbf{前端模块}:位于src/client目录,采用Vue 3组合式API开发
    \begin{itemize}
        \item components/: 通用UI组件库(门店选择器、员工卡片、排班日历等)
        \item stores/: Pinia状态管理(员工状态、门店状态、排班状态)
        \item routes/: 基于角色的路由配置(管理员/店长视图权限分离)
        \item views/: 业务页面(员工管理页、排班调整页、报表页)
    \end{itemize}

    \item \textbf{后端模块}:位于src/server目录,采用微服务架构设计
    \begin{itemize}
        \item services/: 微服务实现(员工服务、门店服务、排班服务)
        \item models/: 数据库模型定义(Sequelize ORM映射)
        \item routes/: RESTful API路由(员工CRUD、排班生成接口)
        \item middleware/: JWT认证、请求日志等中间件
    \end{itemize}

    \item \textbf{算法模块}:位于根目录scheduler.py,包含排班核心逻辑
    \begin{itemize}
        \item SimulatedAnnealing: 模拟退火算法实现类
        \item ScheduleGenerator: 排班方案生成器,用于算法测试
        \item utils/: 包含成本计算、邻域搜索等工具函数
    \end{itemize}

    \item \textbf{公共组件}:
    \begin{itemize}
        \item types/: 共享类型定义(员工实体、排班规则等TypeScript接口)
        \item config/: 统一配置文件(数据库连接、算法参数)
    \end{itemize}
\end{itemize}
代码仓库遵循标准化工程规范,包含ESLint代码检查、Vitest测试框架配置、API文档生成等基础设施。通过package.json和vite.config.ts实现跨环境构建配置,确保开发与生产环境一致性。

\subsection{前端实现}
前端系统基于Vue 3组合式API开发,主要实现以下核心功能模块:

\begin{itemize}
    \item \textbf{员工管理模块}:
    \begin{itemize}
        \item 使用Element Plus表格组件实现员工信息CRUD功能
        \item 开发基于Vuex的状态管理,维护员工数据全局状态
        \item 实现Excel导入导出功能,采用SheetJS库处理文件解析
    \end{itemize}

    \item \textbf{排班可视化模块}:
    \begin{itemize}
        \item 基于FullCalendar开发排班日历组件
        \item 实现拖拽功能,使用HTML5 Drag \& Drop API
        \item 开发实时校验逻辑,监控工时合规性
    \end{itemize}

    \item \textbf{权限控制模块}:
    \begin{itemize}
        \item 实现动态路由加载,根据用户角色过滤路由表
        \item 开发指令级权限控制(v-permission)
    \end{itemize}
\end{itemize}

\subsection{后端实现}
后端采用Node.js + Express框架,核心服务实现如下:

\begin{itemize}
    \item \textbf{员工服务}:
    \begin{itemize}
        \item 实现JWT认证中间件,使用bcryptjs加密密码
        \item 开发RESTful API,支持员工信息的增删改查
        \item 使用Sequelize ORM实现MySQL数据持久化
    \end{itemize}

    \item \textbf{排班服务}:
    \begin{itemize}
        \item 封装模拟退火算法为独立服务
        \item 实现排班规则校验引擎
        \item 开发排班结果缓存机制
    \end{itemize}

    \item \textbf{API网关}:
    \begin{itemize}
        \item 实现请求路由和负载均衡
        \item 开发统一错误处理中间件
        \item 集成Swagger UI自动生成API文档
    \end{itemize}
\end{itemize}

\subsection{数据库设计}
\subsubsection{核心设计原则}
\begin{itemize}
    \item \textbf{范式遵循与数据原子性}
    \begin{itemize}
        \item 采用第三范式(3NF)设计,通过主外键约束保证数据原子性。例如:
        \begin{itemize}
            \item 员工技能表(\texttt{employee\_skills})使用复合主键(\texttt{employee\_id}, \texttt{skill\_id})解耦多值依赖
            \item 技能表(\texttt{skills})字段\texttt{name}定义为不可再分的原子值(如"收银"/"导购")
        \end{itemize}
        
        \item 弱关联引用设计:
        \begin{itemize}
            \item 门店表(\texttt{stores})的\texttt{manager\_id}仅记录\texttt{users.id}数值关联
            \item 通过应用层事务补偿机制确保数据一致性
        \end{itemize}
    \end{itemize}

    \item \textbf{领域模型驱动的模块化架构}
    \begin{table}[H]
        \centering
        \caption{业务模块与表结构映射关系}
        \label{tab:db-modules}
        \begin{tabularx}{\linewidth}{|l|l|X|}
        \hline
        \textbf{业务模块} & \textbf{数据库表} & \textbf{设计要点} \\ \hline
        用户权限 & users & OpenID关联、RBAC角色字段、密码哈希存储 \\ \hline
        组织架构 & stores & 层级结构字段、经理弱关联、门店属性聚合 \\ \hline
        技能体系 & skills + employee\_skills & 技能熟练度分级、多对多关系解耦 \\ \hline
        员工档案 & employees & 工时偏好(最大日/周工时)、岗位属性 \\ \hline
        班次规则 & shifts + shift\_positions & 时间模板、岗位需求动态配置 \\ \hline
        排班执行 & schedules + shift\_assignments & 排班周期管理、人工干预记录 \\ \hline
        \end{tabularx}
    \end{table}

    \item \textbf{性能优化策略}
    \begin{itemize}
        \item \textbf{反规范化设计}:
        \begin{itemize}
            \item 排班记录表(\texttt{shift\_assignments})冗余存储\texttt{store\_id}
            \item 减少通过\texttt{shift\_id}关联查询的开销
        \end{itemize}
        
        \item \textbf{索引优化}:
        \begin{itemize}
            \item 为\texttt{employees(store\_id, position)}创建联合索引
            \item 加速按门店和岗位筛选的高频操作
        \end{itemize}
        
        \item \textbf{扩展性设计}:
        \begin{itemize}
            \item 使用Snowflake算法生成分布式主键
            \item 预留\texttt{schedules\_history}表用于版本追溯
        \end{itemize}
    \end{itemize}
\end{itemize}


\section{结果展示}
\subsection{界面展示}
\begin{figure}[H]
    \centering
    \includegraphics[width=0.8\linewidth]{./source/登录界面.png}
    \caption{登录界面}
    \label{fig:microservice-arch}
\end{figure}
\begin{figure}[H]
    \centering
    \includegraphics[width=0.8\linewidth]{./source/注册界面.png}
    \caption{注册界面}
    \label{fig:microservice-arch}
\end{figure}
\begin{figure}[H]
    \centering
    \includegraphics[width=0.8\linewidth]{./source/主页.png}
    \caption{主页}
    \label{fig:microservice-arch}
\end{figure}
\begin{figure}[H]
    \centering
    \includegraphics[width=0.8\linewidth]{./source/排班规则管理.png}
    \caption{排班规则管理}
    \label{fig:microservice-arch}
\end{figure}




\section{结论}
本文设计并实现了一套基于模拟退火算法的智能排班系统,针对零售行业排班场景中的复杂约束条件,提出了完整的解决方案。系统采用前后端分离架构与微服务技术,实现了员工管理、门店配置、智能排班等核心功能模块。通过模拟退火算法优化排班方案,在保证合规性的同时兼顾员工偏好,显著提升了排班效率。实际测试表明,系统能够快速生成高质量排班方案,有效降低人力成本并提高员工满意度。未来可进一步扩展预测算法精度和多目标优化能力,以适应更复杂的商业场景需求。

% 致谢
\addcontentsline{toc}{section}{致谢}


% 参考文献
\addcontentsline{toc}{section}{参考文献}
\bibliographystyle{gbt7714-numerical} % 中文参考文献样式
\bibliography{references} % 指定参考文献数据库文件

% 附录
\appendix
\addcontentsline{toc}{section}{附录A:系统功能模块详细说明}
% 这里添加附录内容

\addcontentsline{toc}{section}{附录B:核心算法伪代码}
% 这里添加附录内容
\end{document}
