\usepackage{textcomp}    % 提供额外的文本符号
\usepackage{pdfpages}    % 允许将PDF文件作为页面导入到LaTeX文档中
\usepackage{fancyhdr}    % 提供自定义页眉页脚的功能
\usepackage{geometry}    % 用于设置页面尺寸、边距等页面布局参数
\usepackage{titlesec}    % 允许自定义章节标题的格式和样式
\usepackage{fontspec}    % XeLaTeX/LuaLaTeX下的字体设置包,支持系统字体和OpenType特性
\usepackage{xeCJK}       % 用于支持CJK文本的字体设置
\usepackage{amsmath}     % 支持 LaTeX 数学环境
\usepackage{tabularx}
\usepackage{multirow}
\usepackage{caption}
\usepackage{float}

% 定义字体
\setCJKmainfont{Songti SC}  % 调用宋体
\setmainfont{Times New Roman} % 西文主字体

% 定义纸张大小
% 论文采用A4纸打印。页边距:上3厘米,下2厘米,左3厘米,右2厘米;装订线1厘米;页眉距边界2厘米,页脚距边界1厘米。
\geometry{
  a4paper,            % A4纸张标准
  left=3cm,           % 左边距(含装订线)
  right=2cm,          % 右边距
  top=3cm,            % 上边距(页眉距纸张顶部2cm)
  bottom=2cm,         % 下边距(页脚距纸张底部1cm)
  bindingoffset=1cm,  % 装订线(在左边距基础上额外增加)
  headheight=0.8cm,   % 页眉内容区高度
  headsep=0.6cm,      % 页眉与正文间距(2cm页眉边界要求 - 0.8cm页眉高度 - 3cm上边距)
  footskip=1cm        % 页脚底部到页面底部的距离
}

% 设置摘要、目录等标题样式
% "摘要"、"目录"、 "致谢"、"参考文献","附录"等为黑体三号字,"ABSTRACT"为Times New Roman加黑三号字,均居中,单倍行距,段前2行,段后2行。
\renewcommand{\abstractname}{\heiti\zihao{3}\centering 摘要}
\newcommand{\engabstractname}{\selectfont\bfseries\zihao{3}\centering ABSTRACT}
\renewcommand{\contentsname}{\heiti\zihao{3}\centering 目录}
\newcommand{\acknowledgement}{\heiti\zihao{3}\centering 致谢}
\renewcommand{\appendixname}{\heiti\zihao{3}\centering 附录}
\renewcommand{\refname}{\heiti\zihao{3}\centering 参考文献} % 添加参考文献标题样式

% 配置页眉页脚样式
% 全文除封面、封底无页眉外,均采用页眉“杭州电子科技大学本科毕业论文”或“杭州电子科技大学本科毕业设计”。宋体五号字,居中。
% 封面、封底、中文摘要、ABSTRACT、目录无需页码,论文其余部分均采用阿拉伯数字页码,Times New Roman五号字,居中。
\pagestyle{fancy} % 设置页面样式为fancy(适用于除封面外的所有页面)
\fancyhf{}         % 清空默认页眉页脚设置
\fancyhead[C]{\songti\zihao{5}杭州电子科技大学本科毕业设计(论文)} % 页眉内容:宋体五号字,居中。
\renewcommand{\headrulewidth}{0.5pt} % 页眉装饰线粗细设置为0.5磅
% 重定义plain样式
\fancypagestyle{plain}{
  \fancyfoot[C]{\selectfont\zihao{5}\thepage} % Times New Roman五号字,居中。
  \renewcommand{\headrulewidth}{0.5pt} % 保留页眉装饰线
}

% 配置正文内容样式
% 正文内容中文为宋体小四号字,英文为Times New Roman小四号字,行距20磅,标准字符间距。每一章内容均另起一页。
\renewcommand{\normalsize}{\songti\zihao{-4}} % 设置正文默认字体为宋体小四号
\linespread{1.5} % 设置行距为20磅(约为1.5倍行距)
\let\originalsection\section
\renewcommand{\section}{\clearpage\originalsection} % 设置每章另起一页
\setlength{\parindent}{2em} % 设置首行缩进为2个汉字

% 配置正文标题样式
% 正文第一级标题为黑体三号字,居中,单倍行距,段前2行,段后2行。第二级标题为黑体四号字,第三级标题为黑体小四号字
\titleformat{\section}{\heiti\zihao{3}\centering}{\thesection}{1em}{} % 第一级标题为黑体三号字,居中,单倍行距
\titleformat{\subsection}{\heiti\zihao{4}}{\thesubsection}{1em}{} % 第二级标题为黑体四号字
\titleformat{\subsubsection}{\heiti\zihao{-4}}{\thesubsubsection}{1em}{} % 第三级标题为黑体小四号字

