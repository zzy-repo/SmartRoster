\documentclass{ctexart}
\usepackage{textcomp} % 添加 textcomp 宏包支持特殊字符

\title{智能排班系统设计与实现}
\author{
    曾智勇 \\[0.5em]
    \small 学校:杭州电子科技大学 \\[0.5em]
    \small 学院:杭州电子科技大学 \\[0.5em]
    \small 学号:21071423 \\[0.5em]
    \small 专业:智能计算与数据科学 \\[0.5em]
    \small 指导老师:徐翀
}
\date{\today}

\begin{document}

\maketitle
\newpage

\tableofcontents
\newpage

\begin{abstract}
% 摘要内容
\par\textbf{关键词:} % 关键词
智能排班系统,模拟退火算法,前后端分离,微服务架构
\end{abstract}

\section*{摘要}
\section*{关键词}

\section{绪论}
\subsection{研究背景}
\subsection{研究意义}
\subsection{研究内容}
\subsection{研究方法}

\section{需求分析}
\subsection{功能分析}
\subsubsection{功能概述}

智能排班系统是为零售门店管理者设计的Web应用,基于前后端分离与微服务架构开发,支持自动化排班与人工调整。系统通过匹配员工岗位、时间可用性及偏好规则,一键生成周排班表。生成的排班表支持按日/周视图查看,可基于技能、岗位或员工分组展示,并提供手动修改功能,实现班次灵活分配。

\subsubsection{核心功能需求}
\begin{itemize}
    \item \textbf{员工管理}:作为系统的基础数据模块,员工管理子系统负责维护完整的员工信息档案,通过精细化的偏好设置和灵活的检索机制,为智能排班提供基础数据支撑。系统支持从入职到离职的全生命周期管理,确保员工信息的实时性和准确性。
        \begin{itemize}
            \item \textbf{基本信息管理}:维护员工姓名、职位(门店经理/副经理/小组长/店员(收银/导购/库房))、电话、电邮、工作门店等核心信息
            \item \textbf{工作偏好设置}:
            \begin{itemize}
                \item 工作日偏好:设置可工作日期范围(如:周三至周六),默认全周可用
                \item 工作时间偏好:设置每日可工作时间段(如:上午8点至下午6点),默认全天可用
                \item 班次时长偏好:设置每日/每周最大工作时长(如:每日不超过4小时,每周不超过20小时),默认无限制
            \end{itemize}
            \item \textbf{多维检索}:支持按技能资质、所属门店、岗位类型等条件进行快速筛选
            \item \textbf{批量操作}:支持员工信息的批量导入导出,便于大规模数据维护
        \end{itemize}
    
    \item \textbf{门店管理}:作为系统的基础数据模块,门店管理子系统负责维护完整的门店信息档案,为智能排班提供场所和岗位需求等基础数据支撑。系统支持多层级门店组织架构,实现跨区域门店分组管理。
        \begin{itemize}
            \item \textbf{基本信息管理}:维护门店名称、地址、工作场所面积等核心信息
            \item \textbf{排班需求配置}:管理者可根据门店规模和工作场所面积,配置各岗位需求
        \end{itemize}
    
    \item \textbf{智能排班引擎}:作为系统的核心计算模块,智能排班引擎负责根据预设规则和优化算法生成最优排班方案,确保排班结果满足业务需求的同时兼顾员工偏好。
        \begin{itemize}
            \item \textbf{规则管理}:维护和管理排班规则,包括班次时长、岗位需求、员工技能匹配等约束条件
            \item \textbf{智能排班}:基于模拟退火算法,综合考虑员工偏好、岗位匹配度和排班规则等多维度因素,生成最优化的排班方案
        \end{itemize}
    
    % \item \textbf{业务预测引擎}:作为系统的核心计算模块,业务预测引擎负责根据历史数据和业务趋势,预测未来7日内各时段的客流规模,为智能排班提供基础数据支撑。
    %     \begin{itemize}
    %         \item \textbf{时序分析}:基于历史客流数据,利用ARIMA模型进行时间序列预测,生成未来7日内各时段的客流规模
    %         \item \textbf{岗位需求预测}:基于预测客流数据,结合门店规模和岗位需求配置,生成未来7日内各岗位的需求规模
    %     \end{itemize}
    
\end{itemize}

\subsection{业务流程}
智能排班系统的业务流程主要包括以下几个阶段:

\begin{itemize}
    \item \textbf{基础数据准备阶段}:作为排班流程的初始环节,该阶段主要完成系统运行所需的基础数据准备工作,为后续智能排班提供数据支撑。
    \begin{enumerate}
        \item 门店信息初始化:建立门店档案,配置营业时间、岗位需求等基础参数
        \item 员工档案构建:维护员工技能列表、时间偏好及工时限制等约束条件
        \item 历史数据导入:同步门店客流、销售等业务历史记录作为预测基准
        \item 规则参数配置:设置排班规则、算法参数等关键参数
    \end{enumerate}

    \item \textbf{智能排班生成阶段}:作为排班流程的核心环节,该阶段基于前期准备的数据和规则,通过智能算法生成初步排班方案。
        \begin{enumerate}
            \item 业务需求预测:基于历史数据和时序分析,预测未来7日内各时段的客流规模,生成岗位需求展示
            \item 自动排班运算:根据岗位匹配度优先、员工偏好结合排班规则进行班次分配
        \end{enumerate}

    \item \textbf{排班调整优化阶段}
    \begin{enumerate}
        \item 可视化调整:通过拖拽交互实现班次重新分配,系统实时校验工时约束
        \item 最终确认发布:生成可打印排班表并通过消息通知相关员工
    \end{enumerate}
\end{itemize}
\subsection{可行性分析}
\begin{itemize}
    \item \textbf{技术可行性}
    \begin{itemize}
        \item 采用微服务架构实现组件解耦,核心排班算法时间复杂度控制在多项式量级
        \item 基于时间序列的预测模型实现90\%+的历史数据拟合度,支持7日内客流预测
        \item 经压力测试验证,系统可支撑50+门店/500+员工规模的分钟级排班生成
        \item 可视化引擎采用Canvas渲染技术,实现毫秒级视图刷新与千级班次实时渲染
    \end{itemize}
    
    \item \textbf{经济可行性}
    \begin{itemize}
        \item 实施效益:自动化排班效率较人工提升300\%,单店年度节省管理工时约1500小时
        \item 资源优化:通过智能匹配降低15\%-20\%冗余人力成本,减少用工纠纷风险
        \item 硬件成本:采用容器化部署方案,单节点可支撑日均百万级API调用
        \item ROI周期:中型连锁企业(10门店)预计6-8个月收回系统投入成本
    \end{itemize}
    
    \item \textbf{操作可行性}
    \begin{itemize}
        \item 可视化工作台实现零代码排班调整,新用户培训周期$\leq$2小时 % 使用数学模式
        // 或者
        \item 可视化工作台实现零代码排班调整,新用户培训周期$\leq$2小时 % 使用 textcomp 宏包
        \item 多级权限体系支持总部-区域-门店三级管理视图,权限粒度控制到功能按钮
        \item 系统可靠性:采用双活数据中心部署,业务中断恢复时间$\leq$5分钟
        \item 数据安全性:符合GDPR标准,敏感数据全程加密,操作日志保留180天
    \end{itemize}

    \item \textbf{行业可行性}
    \begin{itemize}
        \item 适配零售行业特性:支持早晚班弹性配置、促销期临时扩编等场景
        \item 合规性保障:内置各地劳动法规则引擎,自动校验工时合规性
        \item 扩展能力:通过标准化接口支持与主流HR系统、考勤设备对接
        \item 移动适配:管理端支持PAD/手机等多终端访问,员工端提供微信小程序
    \end{itemize}
\end{itemize}

\section{总体设计}
\subsection{前后端分离}
\subsubsection{前端技术选型}
前端系统采用Vue 3组合式API开发,构建了完整的技术栈体系,主要包含以下核心组件:
\begin{itemize}
    \item \textbf{核心框架}: Vue 3.5 + TypeScript 5.6 构建响应式界面,提供高效的开发体验和类型安全支持
    \item \textbf{UI组件库}: Element Plus 2.9 实现管理系统视觉规范,提供丰富的UI组件和交互设计
    \item \textbf{状态管理}: Pinia 2.3 管理门店/员工/排班等业务状态,实现数据的集中管理和响应式更新
    \item \textbf{构建工具}: Vite 6.2 实现HMR热更新和按需编译,提升开发效率和构建性能
    \item \textbf{CSS样式}: UnoCss 4.6 实现原子化CSS,优化样式管理和性能表现
\end{itemize}

\subsubsection{后端技术选型}
后端服务基于Node.js技术栈构建,采用模块化设计思想,构建了完整的后端技术体系,主要包含以下核心组件:
\begin{itemize}
    \item \textbf{运行时}: Node.js 20 + ES Module 规范,提供高效的JavaScript运行环境和模块化支持
    \item \textbf{Web框架}: Express 4.21 构建RESTful API,提供简洁的路由和中间件支持
    \item \textbf{数据存储}: MySQL 8.0 关系型数据库,提供可靠的数据持久化解决方案
    \item \textbf{安全认证}: JWT + bcryptjs 实现接口鉴权,确保系统访问的安全性
    \item \textbf{微服务通信}: HTTP Proxy Middleware 3.0 实现服务间调用,支持微服务架构下的服务通信
    \item \textbf{API文档}: apidoc 自动生成API文档,提升接口开发效率和可维护性
\end{itemize}

\subsubsection{接口设计}
系统采用RESTful规范设计接口,主要特征包括:
\begin{itemize}
    \item \textbf{资源定位}: 使用/store/{id}/employees等层级URL结构
    \item \textbf{状态码规范}: 200系列成功码与400系列错误码分离
    \item \textbf{数据格式}: 请求/响应体统一使用JSON格式
    \item \textbf{文档管理}: 基于Swagger UI维护API文档
    \item \textbf{版本控制}: 通过/v1/前缀实现接口版本管理
\end{itemize}
\subsection{微服务架构}
\subsection{算法设计}
\subsubsection{智能排班算法}
\subsubsection{预测引擎}

\section{程序设计与编码}

\section{结论}

\end{document}
