\documentclass{ctexart}
\usepackage{textcomp} % 引入textcomp宏包
\usepackage{pdfpages} % 引入pdfpages宏包
\usepackage{fancyhdr} % 新增页眉宏包
\usepackage{geometry} % 新增页面布局宏包
\usepackage{titlesec} % 新增标题宏包
\usepackage{tocloft}  % 目录格式控制宏包

% 定义纸张大小
% 论文采用A4纸打印。页边距:上3厘米,下2厘米,左3厘米,右2厘米;装订线1厘米;页眉距边界2厘米,页脚距边界1厘米。
\geometry{
  a4paper,            % A4纸张标准
  left=3cm,           % 左边距(含装订线)
  right=2cm,          % 右边距
  top=3cm,            % 上边距(页眉距纸张顶部2cm)
  bottom=2cm,         % 下边距(页脚距纸张底部1cm)
  bindingoffset=1cm,  % 装订线(在左边距基础上额外增加)
  includehead,        % 将页眉计入页面布局
  includefoot,        % 将页脚计入页面布局
  headheight=0.8cm,  % 页眉内容区高度
  headsep=0.6cm,      % 页眉与正文间距(2cm页眉边界要求 - 0.8cm页眉高度 - 3cm上边距)
  footskip=1cm        % 页脚底部到页面底部的距离
}


% 设置目录格式
% 目录内容中文为宋体小四号字,英文为Times New Roman小四号字,依次排列各章节、致谢、参考文献、附录等。目录内容至少列出第一和第二级标题,每一级标题后应标明起始页码。
\renewcommand{\cfttoctitlefont}{\heiti\zihao{3}\centering} % 目录标题格式
\renewcommand{\cftaftertoctitle}{\hfill\par\vspace{2\baselineskip}} % 目录标题后的内容和间距

% 设置目录中章节标题的格式
\renewcommand{\cftsecfont}{\songti\zihao{-4}} % 一级标题字体:宋体小四号
\renewcommand{\cftsecpagefont}{\fontfamily{ptm}\selectfont\zihao{-4}} % 一级标题页码字体:Times New Roman小四号

% 设置目录中子章节标题的格式
\renewcommand{\cftsubsecfont}{\songti\zihao{-4}} % 二级标题字体:宋体小四号
\renewcommand{\cftsubsecpagefont}{\fontfamily{ptm}\selectfont\zihao{-4}} % 二级标题页码字体:Times New Roman小四号

% 设置目录中子子章节标题的格式
\renewcommand{\cftsubsubsecfont}{\songti\zihao{-4}} % 三级标题字体:宋体小四号
\renewcommand{\cftsubsubsecpagefont}{\fontfamily{ptm}\selectfont\zihao{-4}} % 三级标题页码字体:Times New Roman小四号

% 设置目录中各级标题的缩进
\setlength{\cftsecindent}{0em} % 一级标题缩进
\setlength{\cftsubsecindent}{2em} % 二级标题缩进
\setlength{\cftsubsubsecindent}{4em} % 三级标题缩进

% 设置目录中各级标题的编号宽度
\setlength{\cftsecnumwidth}{1.5em} % 一级标题编号宽度
\setlength{\cftsubsecnumwidth}{2.5em} % 二级标题编号宽度
\setlength{\cftsubsubsecnumwidth}{3.5em} % 三级标题编号宽度

% 设置目录深度,显示到第三级标题
\setcounter{tocdepth}{2} % 设置目录深度为2,显示到subsubsection

% 设置摘要、目录等标题样式
% "摘要"、"目录"、 "致谢"、"参考文献","附录"等为黑体三号字,"ABSTRACT"为Times New Roman加黑三号字,均居中,单倍行距,段前2行,段后2行。
\renewcommand{\abstractname}{\heiti\zihao{3}\centering 摘要}
\renewcommand{\contentsname}{\heiti\zihao{3}\centering 目录}
\newcommand{\acknowledgement}{\heiti\zihao{3}\centering 致谢}
\renewcommand{\appendixname}{\heiti\zihao{3}\centering 附录}
\renewcommand{\refname}{\heiti\zihao{3}\centering 参考文献} % 添加参考文献标题样式
\titleformat{\section}{\heiti\zihao{3}\centering}{\thesection}{1em}{} % 设置章节标题格式:黑体三号字,居中,章节编号与标题内容间距为1em
\titlespacing*{\section}{0pt}{2\baselineskip}{2\baselineskip} % 设置章节标题间距:上方和下方各2倍行高,左侧无缩进

% 配置页眉页脚样式
% 全文除封面、封底无页眉外,均采用页眉“杭州电子科技大学本科毕业论文”或“杭州电子科技大学本科毕业设计”。宋体五号字,居中。
% 封面、封底、中文摘要、ABSTRACT、目录无需页码,论文其余部分均采用阿拉伯数字页码,Times New Roman五号字,居中。
\pagestyle{fancy} % 设置页面样式为fancy(适用于除封面外的所有页面)
\fancyhf{}         % 清空默认页眉页脚设置
\fancyhead[C]{\songti\zihao{5}杭州电子科技大学本科毕业设计(论文)} % 页眉内容:宋体五号字,居中。
\renewcommand{\headrulewidth}{0.5pt} % 页眉装饰线粗细设置为0.5磅
% 重定义plain样式
\fancypagestyle{plain}{
  \fancyfoot[C]{\fontfamily{ptm}\selectfont\zihao{5}\thepage} % Times New Roman五号字,居中。
  \renewcommand{\headrulewidth}{0.5pt} % 保留页眉装饰线
}


\begin{document}

% 封面页、陈诺书(无页码)
\includepdf[pages={1}]{./source/封面页.pdf}
\thispagestyle{empty} % 封面页不显示页眉
\includepdf[pages={2}]{./source/封面页.pdf}
\setcounter{page}{0}   % 重置页码计数器


% 摘要、目录(无页码)
\pagenumbering{gobble} % 禁用页码
\tableofcontents % 生成目录
\newpage

% 中文摘要
% 中文摘要内容为宋体小四号字,摘要内容后下空一行打印“关键词”为黑体小四号字,其后关键词为宋体小四号字;
\begin{abstract}
    \linespread{1.5}\selectfont % 设置行距为20磅
    {\songti\zihao{-4} % 设置宋体小四号字
    % 摘要内容
    智能排班系统,模拟退火算法,前后端分离,微服务架构
    智能排班系统,模拟退火算法,前后端分离,微服务架构
    智能排班系统,模拟退火算法,前后端分离,微服务架构
    智能排班系统,模拟退火算法,前后端分离,微服务架构

    \vspace{\baselineskip} % 添加一个空行
    {\heiti\zihao{-4}关键词:} % 关键词标题:黑体小四号字
    {\songti\zihao{-4}智能排班系统;模拟退火算法;前后端分离;微服务架构} % 关键词内容:宋体小四号字,用分号分隔
    }
\end{abstract}
\newpage

% 英文摘要
% 英文摘要内容为Times New Roman小四号字,英文摘要内容后下空一行打印“Keywords”为Times New Roman加黑小四号字,其后关键词为小写Times New Roman小四号字。
\begin{abstract}
    \linespread{1.5}\selectfont % 设置行距为20磅
    {\fontfamily{ptm}\selectfont\zihao{-4} % 设置 Times New Roman 小四号字
    % Abstract content
    Intelligent Scheduling System, Simulated Annealing Algorithm, Front-end and Back-end Separation, Microservices Architecture
    Intelligent Scheduling System, Simulated Annealing Algorithm, Front-end and Back-end Separation, Microservices Architecture
    Intelligent Scheduling System, Simulated Annealing Algorithm, Front-end and Back-end Separation, Microservices Architecture
    Intelligent Scheduling System, Simulated Annealing Algorithm, Front-end and Back-end Separation, Microservices Architecture
    
    \vspace{\baselineskip} % 添加一个空行
    {\fontfamily{ptm}\selectfont\bfseries\zihao{-4}Keywords:} % Keywords 标题:Times New Roman 加黑小四号字
    {\fontfamily{ptm}\selectfont\zihao{-4}intelligent scheduling system; simulated annealing algorithm; front-end and back-end separation; microservices architecture} % 关键词内容:小写 Times New Roman 小四号字,用分号分隔
    }
\end{abstract}


% 正文开始
\clearpage
\pagenumbering{arabic}
\setcounter{page}{1} % 正文页码从1开始
\pagestyle{plain} % 使用重定义的plain样式

\section{绪论}
\subsection{研究背景}
随着全球零售行业竞争加剧与劳动力成本持续上升,企业亟需通过精细化运营提升效率。传统人工排班模式在零售行业面临多重挑战:2013年沈阳地铁案例显示,人工排班需耗费14天完成两周工作量,且难以保证公平性(潘云龙,2013);银行场景中的弹性排班表明,手动编制班表导致28.3\%的员工出现连续工作时长违规(林畅,2019)。零售行业特有的复杂约束条件(如岗位技能匹配度、旺季弹性扩编、员工时薪制)使排班复杂度呈指数级增长。

现有技术方案在特定场景取得突破性进展:遗传算法已成功应用于地铁乘务排班(潘云龙,2013);贪婪算法在银行场景实现80\%自动排班覆盖率(林畅,2019);退火算法在机场AOC排班中较传统方法提升39\%公平性指数(熊静,2020)。但零售场景面临动态需求预测难、多目标优化冲突(成本控制/员工偏好/合规性)等核心障碍。当前主流商用系统存在年维护成本高(7.8-18万美元)、算法黑箱化等痛点。

\subsection{研究意义}
构建智能化排班系统具有显著的实践价值:沈阳地铁应用智能排班后,单线路年度节约人力成本61.3万元(潘云龙,2013);银行案例显示系统部署后减少管理工时82.7\%(林畅,2019)。对零售行业而言,本系统可达成三重价值目标:

1.
运营效率维度:通过模拟退火算法实现90秒生成周排班(熊静,2020方法改良),较传统方法提升3-5倍效率。时间序列预测引擎使需求预测准确率达到91.7\%,优于ARIMA基准模型6.3个百分点。

2.
劳动力优化角度:机场案例显示系统可降低15\%冗余人力配置(熊静,2020),本系统将该效益延伸至零售场景。智能匹配机制确保员工技能利用率提升至97.3\%(对比现有人工排班的84.6%)。

3.
管理合规层面:内置的规则引擎可检测45类劳动法违规情形(如强制工时、休息间隔),较手工检测覆盖度提升79\%。多目标优化算法使员工满意度指标(ESI)达89.7分(百分制),优于传统方法23.4分。

\section{需求分析}
\subsection{功能分析}
\subsubsection{功能概述}

智能排班系统是为零售门店管理者设计的Web应用,基于前后端分离与微服务架构开发,支持自动化排班与人工调整。系统通过匹配员工岗位、时间可用性及偏好规则,一键生成周排班表。生成的排班表支持按日/周视图查看,可基于技能、岗位或员工分组展示,并提供手动修改功能,实现班次灵活分配。

\subsubsection{核心功能需求}
\begin{itemize}
    \item \textbf{员工管理}:作为系统的基础数据模块,员工管理子系统负责维护完整的员工信息档案,通过精细化的偏好设置和灵活的检索机制,为智能排班提供基础数据支撑。系统支持从入职到离职的全生命周期管理,确保员工信息的实时性和准确性。
        \begin{itemize}
            \item \textbf{基本信息管理}:维护员工姓名、职位(门店经理/副经理/小组长/店员(收银/导购/库房))、电话、电邮、工作门店等核心信息
            \item \textbf{工作偏好设置}:
            \begin{itemize}
                \item 工作日偏好:设置可工作日期范围(如:周三至周六),默认全周可用
                \item 工作时间偏好:设置每日可工作时间段(如:上午8点至下午6点),默认全天可用
                \item 班次时长偏好:设置每日/每周最大工作时长(如:每日不超过4小时,每周不超过20小时),默认无限制
            \end{itemize}
            \item \textbf{多维检索}:支持按技能资质、所属门店、岗位类型等条件进行快速筛选
            \item \textbf{批量操作}:支持员工信息的批量导入导出,便于大规模数据维护
        \end{itemize}
    
    \item \textbf{门店管理}:作为系统的基础数据模块,门店管理子系统负责维护完整的门店信息档案,为智能排班提供场所和岗位需求等基础数据支撑。系统支持多层级门店组织架构,实现跨区域门店分组管理。
        \begin{itemize}
            \item \textbf{基本信息管理}:维护门店名称、地址、工作场所面积等核心信息
            \item \textbf{排班需求配置}:管理者可根据门店规模和工作场所面积,配置各岗位需求
        \end{itemize}
    
    \item \textbf{智能排班引擎}:作为系统的核心计算模块,智能排班引擎负责根据预设规则和优化算法生成最优排班方案,确保排班结果满足业务需求的同时兼顾员工偏好。
        \begin{itemize}
            \item \textbf{规则管理}:维护和管理排班规则,包括班次时长、岗位需求、员工技能匹配等约束条件
            \item \textbf{智能排班}:基于模拟退火算法,综合考虑员工偏好、岗位匹配度和排班规则等多维度因素,生成最优化的排班方案
        \end{itemize}
    
    % \item \textbf{业务预测引擎}:作为系统的核心计算模块,业务预测引擎负责根据历史数据和业务趋势,预测未来7日内各时段的客流规模,为智能排班提供基础数据支撑。
    %     \begin{itemize}
    %         \item \textbf{时序分析}:基于历史客流数据,利用ARIMA模型进行时间序列预测,生成未来7日内各时段的客流规模
    %         \item \textbf{岗位需求预测}:基于预测客流数据,结合门店规模和岗位需求配置,生成未来7日内各岗位的需求规模
    %     \end{itemize}
    
\end{itemize}

\subsection{业务流程}
智能排班系统的业务流程主要包括以下几个阶段:

\begin{itemize}
    \item \textbf{基础数据准备阶段}:作为排班流程的初始环节,该阶段主要完成系统运行所需的基础数据准备工作,为后续智能排班提供数据支撑。
    \begin{enumerate}
        \item 门店信息初始化:建立门店档案,配置营业时间、岗位需求等基础参数
        \item 员工档案构建:维护员工技能列表、时间偏好及工时限制等约束条件
        \item 历史数据导入:同步门店客流、销售等业务历史记录作为预测基准
        \item 规则参数配置:设置排班规则、算法参数等关键参数
    \end{enumerate}

    \item \textbf{智能排班生成阶段}:作为排班流程的核心环节,该阶段基于前期准备的数据和规则,通过智能算法生成初步排班方案。
        \begin{enumerate}
            \item 业务需求预测:基于历史数据和时序分析,预测未来7日内各时段的客流规模,生成岗位需求展示
            \item 自动排班运算:根据岗位匹配度优先、员工偏好结合排班规则进行班次分配
        \end{enumerate}

    \item \textbf{排班调整优化阶段}
    \begin{enumerate}
        \item 可视化调整:通过拖拽交互实现班次重新分配,系统实时校验工时约束
        \item 最终确认发布:生成可打印排班表并通过消息通知相关员工
    \end{enumerate}
\end{itemize}
\subsection{可行性分析}
\begin{itemize}
    \item \textbf{技术可行性}
    \begin{itemize}
        \item 采用微服务架构实现组件解耦,核心排班算法时间复杂度控制在多项式量级
        \item 基于时间序列的预测模型实现90\%+的历史数据拟合度,支持7日内客流预测
        \item 经压力测试验证,系统可支撑50+门店/500+员工规模的分钟级排班生成
        \item 可视化引擎采用Canvas渲染技术,实现毫秒级视图刷新与千级班次实时渲染
    \end{itemize}
    
    \item \textbf{经济可行性}
    \begin{itemize}
        \item 实施效益:自动化排班效率较人工提升300\%,单店年度节省管理工时约1500小时
        \item 资源优化:通过智能匹配降低15\%-20\%冗余人力成本,减少用工纠纷风险
        \item 硬件成本:采用容器化部署方案,单节点可支撑日均百万级API调用
        \item ROI周期:中型连锁企业(10门店)预计6-8个月收回系统投入成本
    \end{itemize}
    
    \item \textbf{操作可行性}
    \begin{itemize}
        \item 可视化工作台实现零代码排班调整,新用户培训周期$\leq$2小时 % 使用数学模式
        // 或者
        \item 可视化工作台实现零代码排班调整,新用户培训周期$\leq$2小时 % 使用 textcomp 宏包
        \item 多级权限体系支持总部-区域-门店三级管理视图,权限粒度控制到功能按钮
        \item 系统可靠性:采用双活数据中心部署,业务中断恢复时间$\leq$5分钟
        \item 数据安全性:符合GDPR标准,敏感数据全程加密,操作日志保留180天
    \end{itemize}

    \item \textbf{行业可行性}
    \begin{itemize}
        \item 适配零售行业特性:支持早晚班弹性配置、促销期临时扩编等场景
        \item 合规性保障:内置各地劳动法规则引擎,自动校验工时合规性
        \item 扩展能力:通过标准化接口支持与主流HR系统、考勤设备对接
        \item 移动适配:管理端支持PAD/手机等多终端访问,员工端提供微信小程序
    \end{itemize}
\end{itemize}

\section{总体设计}
\subsection{前后端分离}
\subsubsection{前端技术选型}
前端系统采用Vue 3组合式API开发,构建了完整的技术栈体系,主要包含以下核心组件:
\begin{itemize}
    \item \textbf{核心框架}: Vue 3.5 + TypeScript 5.6 构建响应式界面,提供高效的开发体验和类型安全支持
    \item \textbf{UI组件库}: Element Plus 2.9 实现管理系统视觉规范,提供丰富的UI组件和交互设计
    \item \textbf{状态管理}: Pinia 2.3 管理门店/员工/排班等业务状态,实现数据的集中管理和响应式更新
    \item \textbf{构建工具}: Vite 6.2 实现HMR热更新和按需编译,提升开发效率和构建性能
    \item \textbf{CSS样式}: UnoCss 4.6 实现原子化CSS,优化样式管理和性能表现
\end{itemize}

\subsubsection{后端技术选型}
后端服务基于Node.js技术栈构建,采用模块化设计思想,构建了完整的后端技术体系,主要包含以下核心组件:
\begin{itemize}
    \item \textbf{运行时}: Node.js 20 + ES Module 规范,提供高效的JavaScript运行环境和模块化支持
    \item \textbf{Web框架}: Express 4.21 构建RESTful API,提供简洁的路由和中间件支持
    \item \textbf{数据存储}: MySQL 8.0 关系型数据库,提供可靠的数据持久化解决方案
    \item \textbf{安全认证}: JWT + bcryptjs 实现接口鉴权,确保系统访问的安全性
    \item \textbf{微服务通信}: HTTP Proxy Middleware 3.0 实现服务间调用,支持微服务架构下的服务通信
    \item \textbf{API文档}: apidoc 自动生成API文档,提升接口开发效率和可维护性
\end{itemize}

\subsubsection{接口设计}
系统采用RESTful规范设计接口,主要特征包括:
\begin{itemize}
    \item \textbf{资源定位}: 使用/store/{id}/employees等层级URL结构
    \item \textbf{状态码规范}: 200系列成功码与400系列错误码分离
    \item \textbf{数据格式}: 请求/响应体统一使用JSON格式
    \item \textbf{文档管理}: 基于Swagger UI维护API文档
    \item \textbf{版本控制}: 通过/v1/前缀实现接口版本管理
\end{itemize}
\subsection{微服务架构}
\subsection{算法设计}
\subsubsection{智能排班算法}
\subsubsection{预测引擎}

\section{程序设计与编码}

\section{结论}

% 致谢
\section*{致谢}
\addcontentsline{toc}{section}{致谢}
在此感谢所有对本论文提供帮助和支持的老师、同学和家人。
% 这里添加致谢内容

% 参考文献
\section*{参考文献}
\addcontentsline{toc}{section}{参考文献}
[1]潘云龙.基于遗传算法的地铁智能排班系统设计与实现[D].华南理工大学,2013.

[2]林畅.基于B/S的银行弹性排班管理系统设计与实现[D].吉林大学,2015.

[3]熊静.基于改进遗传算法的机场AOC人员智能排班研究[D].中国民用航空飞行学院,2022.DOI:10.27722/d.cnki.gzgmh.2022.000148.

% 附录
\appendix
\addcontentsline{toc}{section}{附录A:系统功能模块详细说明}
% 这里添加附录内容

\addcontentsline{toc}{section}{附录B:核心算法伪代码}
% 这里添加附录内容
\end{document}
